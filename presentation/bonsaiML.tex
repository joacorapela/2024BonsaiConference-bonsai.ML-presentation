% Quick start guide
\documentclass{beamer}

\hypersetup{colorlinks=true}

% \usepackage[colorlinks=true]{hyperref}
\usepackage{natbib}
\usepackage{apalike}
\usepackage{amsthm}
\usepackage{verbatim}

\usetheme{default}
% \usetheme{Madrid}

\newtheorem{probDef}{Definition}
\newtheorem*{probDef*}{Definition}
\newtheorem{claim}{Claim}
\newtheorem*{claim*}{Claim}
\newtheorem*{lemma*}{Lemma}
\newtheorem{probExample}{Example}
\newtheorem{probRule}{Rule}
\newtheorem{probAxiom}{Axiom}
\setbeamertemplate{theorems}[numbered]
\newtheorem{probExercise}{Exercise}

\newtheorem{manualprobRuleinner}{Rule}
\newenvironment{manualProbRule}[1]{%
  \renewcommand\themanualprobRuleinner{#1}%
  \manualprobRuleinner
}{\endmanualprobRuleinner}

\newtheorem{manualprobExampleinner}{Example}
\newenvironment{manualProbExample}[1]{%
  \renewcommand\themanualprobExampleinner{#1}%
  \manualprobExampleinner
}{\endmanualprobExampleinner}

\newcounter{saveenumi}
\newcommand{\seti}{\setcounter{saveenumi}{\value{enumi}}}
\newcommand{\conti}{\setcounter{enumi}{\value{saveenumi}}}
\newcommand{\keepi}{\addtocounter{saveenumi}{-1}\setcounter{enumi}{\value{saveenumi}}}

\setcounter{tocdepth}{1}

\setbeameroption{show notes} % un-comment to see the notes

\DeclareMathOperator*{\argmax}{arg\,max}
\DeclareMathOperator*{\argmin}{arg\,min}

% Title page details
\title{Bonsai.ML}
\subtitle{Intelligent Experimental Control}
\titlegraphic{\includegraphics[width=.5in]{figures/bonsaiMLlogo.png}}
\author{Nicholas Guilbeault and Gon\c{c}alo Lopes and Joaqu\'{i}n Rapela}
\institute{Gatsby Computational Neuroscience Unit\\NeuroGEARS Ltd.}
\date{December 5, 2024}

\AtBeginSection[ ]
{
\begin{frame}{Outline}
    \tableofcontents[currentsection]
\end{frame}
}

\begin{document}

\begin{frame}
	\titlepage
\end{frame}

\begin{frame}{Outline}
    \tableofcontents
\end{frame}

\section{Introduction}

\begin{frame}
    \frametitle{History of Bonsai.ML}

    \begin{itemize}

        \item grant application

        \item developing tools

        \item who cares about Bonsai.ML tools? $\rightarrow$ more focus on
            dissemination

        \item real-time ML

    \end{itemize}
\end{frame}

\begin{frame}
    \frametitle{Goals of Bonsai.ML}

    \begin{itemize}
        \item allow non-programmers use ML tools,
        \item learn from non-programmers what ML tools are useful for them
    \end{itemize}
\end{frame}

\begin{frame}
    \frametitle{Need for Real-Time and Reactive ML}

    Conventional ML operates on stored datasets. We need real-time ML that
    operates on infinite data stream with time-varying statistical properties.

    If a sensor fails, our inferences need to continue. That is, we need
    reactive ML (e.g., \href{https://rxinfer.ml/}{rx.infer}).
\end{frame}

\begin{frame}
    \frametitle{Bonsai.ML demos}

\end{frame}

\subsection{How Bonsai.ML started}

\begin{frame}
    \frametitle{AEON project}

    We are building a new type of experimentation:
    continual recording of behavioural and neural data of mice foraging in large
    arenas for weeks to months.

    \begin{center}
        \includegraphics[width=2in]{figures/foragingMouse.png}
        \includegraphics[width=2in]{figures/mouseOnWheel.png}
    \end{center}

\end{frame}

\begin{frame}
    \frametitle{BBSRC grant: Machine learning for Neuroscience experimental
    control}

    \textcolor{red}{Abstract}

    \tiny
    To understand the brain, scientists aim to explain how animal behaviour
    relates to neural activity. This requires the design and precise control of
    behavioural experiments, wherein animals perform particular tasks while
    experimenters either record or manipulate neural activity in specific
    neural circuits. Such experiments require data acquisition software that
    integrates and controls hardware from multiple recording devices (cameras,
    electrodes, sensors), and analysis tools that can interpret large and
    complex datasets. Progress is held back by the lack of standardised tools
    for design and implementation of experimental protocols, and the difficulty
    of integrating state-of-the-art data processing and neuroinformatics into
    custom experimental designs. The fields of behavioural and brain sciences
    have consequently suffered from both inefficiency and poor reproducibility,
    due to disparate data acquisition and analysis solutions created
    independently across laboratories. To address these challenges, we propose
    to extend, enhance, maintain and support \textbf{Bonsai}, a fully
    integrated software environment to enable cutting-edge reproducible systems
    neuroscience experiments using animal models, with a particular emphasis on
    machine-intelligence-enabled, real-time neuroinformatics methods. While
    Bonsai is already adopted by hundreds of scientists worldwide, we aim to
    extend Bonsai's functionality with a toolbox of online and offline Machine
    Intelligence tools for analysis of behavioural and neural data (video-based
    analysis of behavioural motifs, real-time and offline analysis of neural
    signals), and create an open-access platform for software sharing and
    integration with multiple programming languages. Enhancing Bonsai's
    ecosystem will be a game-changer for behavioural and brain science
    experiments by enabling new types of research, increasing and diversifying
    user base, and dramatically improve efficiency and reproducibility of
    research.

    \hfill\href{https://gow.bbsrc.ukri.org/grants/AwardDetails.aspx?FundingReference=BB\%2FW019132\%2F1}{grant
    details}

\end{frame}

\note[itemize]{
    \item at the time of the grant I did not know much about Bonsai, and I did not want to know about it because it was a Windows software
    \item but the grant reviewers helped me understand how good Bonsai was :)
    \item controlling experiments using advanced machine learning methods
        \begin{itemize}
            \item e.g., testing causality of neural patterns on behaviour; deactivate a brain region when you forecast the appearance of a pattern of brain activity related to a given behaviour, and check if after the deactivation the behavioural pattern disappears
            \item  e.g., only visually stimulate an animal until you achieve a certain precision in the receptive field estimation
        \end{itemize}
}

\begin{frame}
    \frametitle{My experience in machine learning and Neuroscience}

    I have developed methods for:

    \begin{itemize}

        \item nonlinear regression methods to estimate receptive fields of
            visual cells~\citep{rapelaEtAl06,rapelaEtAl10},

        \item Bayesian linear regression methods to understand the relation
            between phase concentration in the human EEG and
            attention~\citep{rapelaEtAl12-attentionSwitch,rapelaEtAl12-eyeTracking,rapelaEtAl18-avshift},

        \item dynamical systems models to model the relation between ECoG
            measurements and speech production in
            humans~\citep{rapelaInPrepTWsInSpeech,rapelaInPrepSyncTWs,rapelaInPrepSyncTWsII},

        \item unsupervised models to characterise epilepsy using Utah array
            recordings from
            humans~\citep{rapelaEtAl19-epilepsy-tsne,rapelaAndTodorov19-epilepsy-hmm}.

    \end{itemize}

    and I am the main developer of
    \href{https://github.com/joacorapela/svGPFA}{svGPFA}, a method using
    variational inference on Gaussian processes to infer latent variables from
    Neuropixels population recordings.

\end{frame}

\begin{frame}
    \frametitle{What type of machine learning we want for Bonsai?}

    \begin{description}

        \item[supervised, unsupervised and reinforcement methods]\mbox{}\\

            \begin{description}

                \item[supervised methods] find mappings between inputs X and outputs y, like in curve fitting

                \item[unsupervised methods] discover structure in inputs X, without any output, like in clustering

                \item[reinforcement learning methods] learn relations between inputs X and actions a to maximise future rewards.

            \end{description}

        \item[batch vs online processing]\mbox{}\\

            \begin{description}

                \item[batch processing] all data is read (generally from files)
                    and processed at the same time.

                \item[online processing] data is processes as it arrives and
                    processed one at a time

            \end{description}

    \end{description}

\end{frame}

\begin{frame}
    \frametitle{What type of machine learning we want for Bonsai?}

    \begin{description}

        \item[stationary vs non-stationary data]\mbox{}\\

            \begin{description}

                \item[stationary data] has statistics (i.e., characteristics)
                    that do not change with time.

                \item[non-stationary data] has time-varying statistics.

            \end{description}

        \item[iid vs time-series datasets]\mbox{}\\

            \begin{description}

                \item[iid datasets] contain samples that are unrelated (i.e., independent) from each other and all come from the same distribution. For example a dataset of coin tosses is iid.

                \item[time-series datasets] contain samples that are related to each other. For example a dataset of frames from a movie is a time-series one.

            \end{description}

    \end{description}

\end{frame}

\begin{frame}
    \frametitle{What type of machine learning we want for Bonsai?}

    \begin{description}

        \item[probabilistic vs deterministic models]\mbox{}\\

            \begin{description}

                \item[probabilistic models] assume that variables of interest are random
      quantities, and seek to estimate their distribution.

                \item[deterministic models]treat their variables of interest as deterministic
      quantities.

            \end{description}

  \item[reactive vs non-reactive models]\mbox{}\\

            \begin{description}

            \item[non-reactive inference models] perform inference by following a pre-established sequence of steps, assuming availability of data before each step begins.

            \item[reactive inference models] react to data availability performing a inference step only when data becomes available. Reactive inference models allow continuous inference in scenarios were data sources (e.g., cameras) can appear or disappear over time.

            \end{description}

    \end{description}

\end{frame}

\begin{frame}
    \frametitle{Machine learning models for Bonsai}

    For Bonsai we want ML models that:

    \begin{description}

        \item[can process online data] process one data item at a time, when they are produced (reactively), and can handle infinite data streams

        \item[are non-stationary] can process data with time-varying statistics

        \item[can handle time-series datasets] as most neuroscience datasets are time series (e.g., behavioural videos, neuron spike counts).

        \item[are reactive] can continue doing inference when adding/removing data sources

    \end{description}

\end{frame}

\begin{frame}
    \frametitle{Large variety of machine learning models}

    \begin{center}
        \includegraphics[width=4.5in]{figures/ml_map.png}
    \end{center}

\end{frame}

\begin{frame}
    \frametitle{Our choices}

    We decided to focus on:

    \begin{itemize}

        \item neuro applications,

        \item Bayesian probabilistic models.

    \end{itemize}

    \onslide<2-> {
    Shall we distribute machine learning software:

    \begin{itemize}

        \item addressing a single need (e.g., deeplabcut for tracking body
            parts, or MOSEQ for inferring behavioural syllables), or

        \item generic ML software addressing multiple needs (e.g., linear
            dynamical systems that can be use to infer kinematics, or to
            discover neural latent variables from population recordings)?

    \end{itemize}

    We opted for the latter.
    }

\end{frame}

\begin{frame}
    \frametitle{Current Bonsai.ML software}

    We built:

    \begin{itemize}

        \item linear dynamical systems to characterise kinematics of foraging
            mice,

        \item hidden Markov models to estimate discrete behavioural states
            (i.e., behavioural syllables) of foraging mice.

    \end{itemize}

    \onslide<2->{
    We are now working on:

    \begin{itemize}

        \item online Bayesian linear regression models to, for example, estimate receptive
            fields of visual neurons,

        \item mark-based hippocampal decoding methods to characterise replay in
            rodents, without spike sorting.

    \end{itemize}
    }

\end{frame}

\begin{frame}
    \frametitle{A new type of machine learning for Bonsai}

    We are excited about working on a new type of machine learning for Bonsai.
    One that:

    \begin{itemize}

        \item process datastreams online,

        \item allows non-stationary datastreams,

        \item is reactive.

    \end{itemize}

    We trust that this new type of machine learning will uncover new findings
    on behaviour and brain function.

\end{frame}


\section{Linear Regression}

\begin{frame}
    \frametitle{Linear Regression: Fundamental Concepts}

    \begin{itemize}
        \item main concepts of linear regression
        \item Online Bayesian Linear Regression (can process infinite data streams, but assumes stationarity)
        \item Recursive least squares (can process infinite data streams, and
            does not assume stationarity)
    \end{itemize}

\end{frame}

\begin{frame}
    \frametitle{Linear Regression: Practical}

    Estimation of receptive fields of visual cells from the Allen Institute.

\end{frame}

\begin{frame}
    \frametitle{Linear regression example}

	\begin{center}
		\includegraphics[width=2.2in]{figures/visVesIntegration.png}
	\end{center}
	\hfill\href{https://www.biorxiv.org/content/10.1101/2021.01.22.427789v4.abstract}{Keshavarzi et al., 2021}
	\begin{columns}
		\onslide<2->{
		\begin{column}{0.5\textwidth}
			\begin{center}
				\includegraphics[width=1.5in]{figures/spikeRateVsabsSpeedV1VisVes.png}
			\end{center}
		\end{column}
		}
		\onslide<3->{
		\begin{column}{0.5\textwidth}
			\textcolor{blue}{Is there a linear relation between the speed of rotation and the firing rate of visual cells?}
		\end{column}
		}
	\end{columns}
\end{frame}

\begin{frame}
    \frametitle{Estimating nonlinear receptive fields from natural images}

    \href{https://jov.arvojournals.org/article.aspx?articleid=2192869}{Rapela et al., 2006}.

\end{frame}

\begin{frame}
    \frametitle{Linear regression model}

	\scriptsize
	\begin{description}
		\item[simple linear regression model]
			\begin{align*}
                y(x_i, \mathbf{w})&=w_0+w_1x_i
				                   =\raisebox{0.60em}{$[1,x_i]$}
									\left[\begin{array}{c}
								        w_0\\
								        w_1
								    \end{array}\right]
				                   =\raisebox{0.60em}{$[\phi_0(x_i),\phi_1(x_i)]$}
									\left[\begin{array}{c}
								        w_0\\
								        w_1
								    \end{array}\right]\\
                                  & =
									\boldsymbol{\phi}(x_i)^\intercal\mathbf{w}
			\end{align*}
		\item[polynomial regression model]
			\begin{align*}
				y(x_i, \mathbf{w})&=w_0+w_1x_i+w_2x_i^2+w_3x_i^3
				                   =\raisebox{1.80em}{$[1,x_i,x_i^2,x_i^3]$}
									\left[\begin{array}{c}
								        w_0\\
								        w_1\\
								        w_2\\
								        w_3
								    \end{array}\right]\\
				                  &=\raisebox{1.80em}{$[\phi_0(x_i),\phi_1(x_i),\phi_2(x_i),\phi_3(x_i)]$}
									\left[\begin{array}{c}
								        w_0\\
								        w_1\\
								        w_2\\
								        w_3
								    \end{array}\right]=
									\boldsymbol{\phi}(x_i)^\intercal\mathbf{w}
			\end{align*}
		\item[basis functions linear regression model]
			\begin{align*}
				y(x_i, \mathbf{w})&=\boldsymbol{\phi}(x_i)^\intercal\mathbf{w}=\sum_{j=1}^Mw_j\phi_j(x_i)
			\end{align*}
	\end{description}
	\normalsize
\end{frame}

\begin{frame}
    \frametitle{Linear regression model}

    \scriptsize
    \begin{align*}
        \mathbf{y}(\mathbf{x},\mathbf{w})&=
            \left[\begin{array}{c}
                      y(x_1,\mathbf{w})\\
                      y(x_2,\mathbf{w})\\
                      \ldots\\
                      y(x_N,\mathbf{w})
                  \end{array}\right]=
            \left[\begin{array}{cccc}
                      \phi_1(x_1)&\phi_2(x_1)&\ldots&\phi_M(x_1)\\
                      \phi_1(x_2)&\phi_2(x_2)&\ldots&\phi_M(x_2)\\
                      \vdots     &\vdots     &\ldots&\vdots\\
                      \phi_1(x_N)&\phi_2(x_N)&\ldots&\phi_M(x_N)
            \end{array}\right]\left[\begin{array}{c}
                                        w_1\\
                                        w_2\\
                                        \vdots\\
                                        w_M
                                    \end{array}\right]\\
                                         &=\boldsymbol{\Phi}\mathbf{w}
    \end{align*}

    where
    $\mathbf{y}(\mathbf{x},\mathbf{w})\in\mathbb{R}^N,\boldsymbol{\Phi}\in\mathbb{R}^{N\times M},\mathbf{w}\in\mathbb{R}^M$.
    \normalsize
\end{frame}

\begin{frame}
    \frametitle{Basis functions for regression}

		\begin{center}
			\includegraphics[width=3.5in]{figures/basisFunctions.png}
		\end{center}
        \hfill\scriptsize\citet{bishop06}

        \begin{description}
            \item[polynomial] $\phi_i(x)=x^i$
            \item[Gaussian] $\phi_i(x)=\exp(-\frac{(x-\mu_i)^2}{2\sigma^2})$
            \item[sigmoidal]
                $\phi_i(x)=\frac{1}{1+\exp(-\frac{x-\mu_i}{\sigma^2})}$
        \end{description}
\end{frame}

\begin{frame}
    \frametitle{Example dataset}

    \begin{center}
        \includegraphics[width=4in]{figures/exampleDataset.png}
    \end{center}
    \hfill\scriptsize\citet{bishop06}

\end{frame}

\subsection{Least-squares regression}

\begin{frame}
    \frametitle{Least-squares estimation of model parameters
    \citep{trefethenAndBau97}}

    \scriptsize
    \begin{probDef}[Least-squares problem]
        Given $\boldsymbol{\Phi}\in\mathbb{R}^{N\times M},N\ge
        M,\mathbf{t}\in\mathbb{R}^N$, find $\mathbf{w}\in\mathbb{R}^M$ such
        that $E_{LS}(\mathbf{w})=||\mathbf{t}-\boldsymbol{\Phi}\boldsymbol{w}||_2$ is minimised.
    \end{probDef}
    \begin{theorem}[Least-squares solution]
        Let $\boldsymbol{\Phi}\in\mathbb{R}^{N\times M} (N\ge M)$ and
        $\mathbf{t}\in\mathbb{R}^N$ be given. A vector
        $\mathbf{w}\in\mathbb{R}^M$ minimises
        $||\mathbf{r}||_2=||\mathbf{t}-\boldsymbol{\Phi}\mathbf{w}||_2$, thereby solving the
        least-squares problem, if and only if
        $\mathbf{r}\perp\text{range}(\boldsymbol{\Phi})$, that is,
        $\boldsymbol{\Phi}^\intercal\mathbf{r}=0$,
        or equivalently,
        $\boldsymbol{\Phi}^\intercal\boldsymbol{\Phi}\mathbf{w}=\boldsymbol\Phi^\intercal\mathbf{t}$,
        or again equivalently,
        $P\mathbf{t}=\boldsymbol{\Phi}\mathbf{w}$,
        where $P\in\mathbf{R}^{N\times N}$ is the orthogonal projector onto
        $\text{range}(A)$ (i.e., $P=A\;(A^\intercal A)^{-1}A^\intercal)$.

    \end{theorem}
	\begin{center}
		\includegraphics[width=4in]{figures/leastSquares.png}
	\end{center}
    \hfill\scriptsize\citet{bishop06}
    \normalsize

    \note{
    Given a set of $N$ observations, $\mathbf{t}$, $N>M$, we want to find model
    parameters $\mathbf{w}$ such that the model outputs,
    $\mathbf{t}(\mathbf{x},\mathbf{w})$ equal the observations. This is
    generally impossible, because the degrees of freedom of the observations,
    $N$, is generally larger than the degrees of freedom of the model
    $\mathbf{t}(\mathbf{x},\mathbf{w})$, $M$. We instead solve the following
    least-squares problem.
    }
\end{frame}

\begin{frame}[fragile]
    \frametitle{Instruction to run notebooks in Google Colab}

    \begin{enumerate}
        \item open a notebook from
            \href{https://github.com/joacorapela/gcnuBridging2023/tree/master/docs/sphinx/build/html/notebooks/auto_examples/bayesianLinearRegression}{here}
        \item replace \textbf{github.com} by \textbf{githubtocolab.com} in the
            URL
        \item insert a cell at the beginning of the notebook with the following
            content
        \seti
    \end{enumerate}

    \tiny
    \begin{verbatim}
       !git clone https://github.com/joacorapela/gcnuBridging2023.git
       %cd gcnuBridging2023
       !pip install -e .
    \end{verbatim}
    \normalsize

    \begin{enumerate}
        \conti
        \item from the menu \textbf{Runtime} select \textbf{Run all}.
    \end{enumerate}
\end{frame}

\begin{frame}
    \frametitle{Code for least-squares estimation of model parameters}

    \begin{itemize}
        \item \href{https://joacorapela.github.io/gcnuBridging2023/auto\_examples/bayesianLinearRegression/plotOverfittingLeastSquares.html\#sphx-glr-auto-examples-bayesianlinearregression-plotoverfittingleastsquares-py}{overfitting}
        \item \href{https://joacorapela.github.io/gcnuBridging2023/auto\_examples/bayesianLinearRegression/plotCrossValidationLeastSquares.html\#sphx-glr-auto-examples-bayesianlinearregression-plotcrossvalidationleastsquares-py}{cross validation}
        \item \href{https://joacorapela.github.io/gcnuBridging2023/auto\_examples/bayesianLinearRegression/plotLackOfOverfittingInLeastSquaresForLargerDatasetSize.html\#sphx-glr-auto-examples-bayesianlinearregression-plotlackofoverfittinginleastsquaresforlargerdatasetsize-py}{larger datasets allow more complex models}
    \end{itemize}

\end{frame}

\begin{frame}
    \frametitle{Regularised least-squares estimation of model parameters}

    To cope with the overfitting of least squares, we can add to the least
    squares optimisation criterion a term that enforces coefficients to be
    zero. The regularised least-squares optimisation criterion becomes:

    \begin{align*}
        E_{RLS}(\mathbf{w})=||\mathbf{t}-\boldsymbol{\Phi}\mathbf{w}||_2^2+\lambda||\mathbf{w}||_2^2
    \end{align*}

    where $\lambda$ is the regularisation parameter that weights the strength
    of the regularisation.
\end{frame}

\begin{frame}
    \frametitle{Regularised least-squares estimation of model parameters}
	\scriptsize
	\begin{claim}[Regularised least-squares estimate]
		\begin{align*}
			\mathbf{w}_{RLS}=\argmin_{\mathbf{w}}E_{RLS}(\mathbf{w})=\argmin_{\mathbf{w}}||\mathbf{t}-\boldsymbol{\Phi}\mathbf{w}||_2^2+\lambda||\mathbf{w}||_2^2=(\lambda\mathbf{I}+\boldsymbol{\Phi}^\intercal\boldsymbol{\Phi})^{-1}\boldsymbol{\Phi}^\intercal\mathbf{t}
		\end{align*}
	\end{claim}
	\tiny
	\begin{proof}
		Since $E_{RLS}(\mathbf{w})$ is a polynomial of order two on the elements of $\mathbf{w}$ (i.e., a quadratic form), we can use the \emph{Completing the Squares} technique below to find its minimum.
		\begin{align}
			\boldsymbol{\mu}&=\argmax_{\mathbf{w}}\mathcal{N}(\mathbf{w}|\boldsymbol{\mu},\Sigma)=\argmax_{\mathbf{w}}\log\mathcal{N}(\mathbf{w}|\boldsymbol{\mu},\Sigma)\nonumber\\
                            &=\argmax_{\mathbf{w}}\{K-\frac{1}{2}(-2\boldsymbol{\mu}^\intercal\Sigma^{-1}\mathbf{w}+\mathbf{w}\Sigma^{-1}\mathbf{w})\}\label{eq:completingTheSquaresStep2}\\
                            &=\argmin_{\mathbf{w}}\{-K+\frac{1}{2}(-2\boldsymbol{\mu}^\intercal\Sigma^{-1}\mathbf{w}+\mathbf{w}\Sigma^{-1}\mathbf{w})\}\nonumber\\
                            &=\argmin_{\mathbf{w}}\{K_1-2\boldsymbol{\mu}^\intercal\Sigma^{-1}\mathbf{w}+\mathbf{w}\Sigma^{-1}\mathbf{w}\}\label{eq:completingTheSquares}
		\end{align}

		% Note: Eq.~\ref{eq:completingTheSquaresStep2} uses Eq.~\ref{eq:gaussianQuadratic}.

		To find the minimum of a quadratic form, we write it in the form of the
terms inside the curly brackets of Eq.~\ref{eq:completingTheSquares}, and the
term corresponding to $\boldsymbol{\mu}$ will be the minimum.

		\phantom\qedhere
	\end{proof}
	\normalsize
\end{frame}

\begin{frame}
    \frametitle{Regularised least-squares estimation of model parameters}
	\tiny
		\begin{proof}
			Let's write $E_{RLS}$ in the form of the terms inside the curly brackets of Eq.~\ref{eq:completingTheSquares}.

		\begin{align*}
			E_{RLS}&=||\mathbf{t}-\boldsymbol{\Phi}\mathbf{w}||_2^2+\lambda||\mathbf{w}||_2^2=(\mathbf{t}-\boldsymbol{\Phi}\mathbf{w})^\intercal(\mathbf{t}-\boldsymbol{\Phi}\mathbf{w})+\lambda\mathbf{w}^\intercal\mathbf{w}\\
                   &=\mathbf{t}^\intercal\mathbf{t}-2\mathbf{t}^\intercal\boldsymbol{\Phi}\mathbf{w}+\mathbf{w}^\intercal\boldsymbol{\Phi}^\intercal\boldsymbol{\Phi}\mathbf{w}+\lambda\mathbf{w}^\intercal\mathbf{w}\\
                   &=\mathbf{t}^\intercal\mathbf{t}-2\mathbf{t}^\intercal\boldsymbol{\Phi}\mathbf{w}+\mathbf{w}^\intercal(\boldsymbol{\Phi}^\intercal\boldsymbol{\Phi}+\lambda\mathbf{I}_M)\mathbf{w}
		\end{align*}
		Calling
		\begin{align*}
			\Sigma^{-1}&=\boldsymbol{\Phi}^\intercal\boldsymbol{\Phi}+\lambda\mathbf{I}_M\\
			\boldsymbol{\mu}^\intercal\Sigma^{-1}&=\mathbf{t}^\intercal\boldsymbol{\Phi}\;\text{or}\;\boldsymbol{\mu}^\intercal=\mathbf{t}^\intercal\boldsymbol{\Phi}\Sigma\;\text{or}\;\boldsymbol{\mu}=\Sigma\boldsymbol{\Phi}^\intercal\mathbf{t}=\left(\boldsymbol{\Phi}^\intercal\boldsymbol{\Phi}+\lambda\mathbf{I}_M\right)^{-1}\boldsymbol{\Phi}^\intercal\mathbf{t}
		\end{align*}
		we can express
		\begin{align*}
			E_{RLS}=K+2\boldsymbol{\mu}^\intercal\Sigma^{-1}\mathbf{w}+\mathbf{w}\Sigma^{-1}\mathbf{w}
		\end{align*}
		Then
		\begin{align*}
			 \mathbf{w}_{RLS}=\argmin_{\mathbf{w}}E_{RLS}(\mathbf{w})=\boldsymbol{\mu}=\left(\boldsymbol{\Phi}^\intercal\boldsymbol{\Phi}+\lambda\mathbf{I}_M\right)^{-1}\boldsymbol{\Phi}^\intercal\mathbf{t} 
		\end{align*}
		\end{proof}
	\normalsize
\end{frame}

\begin{frame}
    \frametitle{Code for regularised least-squares estimation of model parameters}
    \begin{itemize}
        \item \href{file:///nfs/ghome/live/rapela/dev/teaching/gcnuBridging2023/repo/docs/sphinx/build/html/auto\_examples/bayesianLinearRegression/plotRegularizedLeastSquares.html\#sphx-glr-auto-examples-bayesianlinearregression-plotregularizedleastsquares-py}{control of overfitting}
    \end{itemize}
\end{frame}

\subsection{Maximum-likelihood regression}

\begin{frame}
    \frametitle{Maximum-likelihood estimation of model parameters}

	\scriptsize
    \begin{probDef}[Likelihood function]
        For a statistical model characterised by a probability density
        function $p(\mathbf{x}|\theta)$ (or probability mass function
        $P_\theta(X=\mathbf{x})$) the likelihood function is a function of the
        parameters $\theta$, $\mathcal{L}(\theta)=p(\mathbf{x}|\theta)$
        (or $\mathcal{L}(\theta)=P_\theta(\mathbf{x})$).
   \end{probDef}

    \begin{probDef}[Maximum likelihood parameters estimates]
        The maximum likelihood parameters estimates are the parameters that
        maximise the likelihood function.

        \begin{align*}
            \theta_{ML}=\argmax_{\theta}\mathcal{L}(\theta)
        \end{align*}
   \end{probDef}

	\normalsize

\end{frame}

\begin{frame}
    \frametitle{Maximum-likelihood estimation for the basis function linear
    regression model}

    \footnotesize
    We seek the parameter $\mathbf{w}_{ML}$ and $\beta_{ML}$ that maximised the following likelihood function

    \begin{align}
        \mathcal{L}(\mathbf{w},\beta)=p(\mathbf{t}|\mathbf{w},\beta)=\mathcal{N}(\mathbf{t}|\boldsymbol{\Phi}\mathbf{w},\beta^{-1}I_N)\label{eq:likelihoodLinearRegression}
    \end{align}

    They are

    \begin{align}
        \mathbf{w}_{ML}&=(\boldsymbol{\Phi}^\intercal\boldsymbol{\Phi})^{-1}\boldsymbol{\Phi}^\intercal\mathbf{t}\label{eq:wML}\\
        \frac{1}{\beta_{ML}}&=\frac{1}{N}||\mathbf{t}-\boldsymbol{\Phi}\mathbf{w}_{ML}||_2^2\label{eq:betaML}
    \end{align}

	\begin{itemize}

		\item first regression method that assumes random observations

		\item if the likelihood function is assumed to be Normal,
		maximum-likelihood and least-squares coefficients estimates are equal.

	\end{itemize}

    \normalsize

\end{frame}

\begin{frame}
    \frametitle{Maximum likelihood: exercise}

    \scriptsize
    \begin{probExercise}
        Derive the formulas for the maximum likelihood estimates of the
        coefficients, $\mathbf{w}$, and noise precision, $\beta$, of the basis
        functions linear regression model given in Eqs.~\ref{eq:wML}
        and~\ref{eq:betaML}.
    \end{probExercise}

    \tiny
    \begin{proof}[Solution]
        \begin{align*}
            \mathcal{L}(\mathbf{w},\beta)&=p(\mathbf{t}|\mathbf{w},\beta)=\mathcal{N}(\mathbf{t}|\boldsymbol{\Phi}\mathbf{w},\beta^{-1}\mathbf{I})\\
                                         &=\frac{1}{(2\pi)^\frac{N}{2}|\beta^{-1}\mathbf{I}|^\frac{1}{2}}\exp\left(-\frac{\beta}{2}||\mathbf{t}-\boldsymbol{\Phi}\mathbf{w}||_2^2\right)\\
            \log\mathcal{L}(\mathbf{w},\beta)&=-\frac{N}{2}\log{2\pi}+\frac{N}{2}\log\beta-\frac{\beta}{2}||\mathbf{t}-\boldsymbol{\Phi}\mathbf{w}||_2^2\\
            \mathbf{w}_{ML}&=\argmax_{\mathbf{w}}\log\mathcal{L}(\mathbf{w},\beta)=\argmin_{\mathbf{w}}||\mathbf{t}-\boldsymbol{\Phi}\mathbf{w}||_2^2=(\boldsymbol{\Phi}^\intercal\boldsymbol{\Phi})^{-1}\boldsymbol{\Phi}^\intercal\mathbf{t}\\
            \frac{\partial}{\partial\beta}\log p(\mathbf{t}|\mathbf{w}_{ML},\beta)&=\frac{N}{2}\frac{1}{\beta}-\frac{1}{2}||\mathbf{t}-\boldsymbol{\Phi}\mathbf{w}_{ML}||_2^2\\
            \frac{\partial}{\partial\beta}\log
            p(\mathbf{t}|\mathbf{w}_{ML},\beta_{ML})&=0\quad\text{iff}\quad\frac{1}{\beta_{ML}}=\frac{1}{N}||\mathbf{t}-\boldsymbol{\Phi}\mathbf{w}_{ML}||_2^2
        \end{align*}
		\phantom\qedhere
    \end{proof}
    \normalsize
\end{frame}

\subsection{Bayesian linear regression}

\begin{frame}
    \frametitle{Bayesian linear regression: motivation}

	\begin{itemize}
		\item elegant,
		\item naturally leads to online regression,
		\item does not require cross-validation for model selection,
		\item it is the first step to more complex Bayesian modelling.
	\end{itemize}

\end{frame}

\subsubsection{Batch Bayesian linear regression}

\begin{frame}
    \frametitle{Batch Bayesian linear regression: posterior distribution of parameters}

	\scriptsize
	In Bayesian linear regression we seek the posterior distribution of the
weights of the linear regression model, $\mathbf{w}$, given the observations, which
is proportional to the product of the likelihood function,
$p(\mathbf{t}|\mathbf{w})$, and the prior, $p(\mathbf{w})$; i.e., 

	\begin{align}
		p(\mathbf{w}|\mathbf{t})\propto
        p(\mathbf{t}|\mathbf{w})p(\mathbf{w})\label{eq:priorLinearRegression}
	\end{align}

	To calculate this posterior below we use the likelihood function defined in
Eq.~\ref{eq:likelihoodLinearRegression} and the following prior

	\begin{align*}
		p(\mathbf{w})=\mathcal{N}(\mathbf{w}|\mathbf{0},\alpha^{-1}\mathbf{I})
	\end{align*}

	Using the expression of the conditional of the Linear Gaussian model,
Eq.~\ref{eq:conditionalLinearGaussianModel}, we obtain

	\begin{align}
		p(\mathbf{w}|\mathbf{t})&=\mathcal{N}(\mathbf{w}|\mathbf{m}_N,\mathbf{S}_N)\nonumber\\
		\mathbf{m}_N&=\beta\mathbf{S}_N\boldsymbol{\Phi}^\intercal\mathbf{t}\label{eq:blrPosteriorMean}\\
		\mathbf{S}_N^{-1}&=\alpha\mathbf{I}+\beta\boldsymbol{\Phi}^\intercal\boldsymbol{\Phi}\label{eq:blrPosteriorCov}
	\end{align}

	\normalsize
\end{frame}

\begin{frame}
    \frametitle{Batch Bayesian linear regression: exercise}

    \scriptsize
    \begin{probExercise}
		Derive the formulas for the Bayesian posterior mean
(Eq.~\ref{eq:blrPosteriorMean}) and covariance (Eq.~\ref{eq:blrPosteriorCov})
of the basis function linear regression model.
    \end{probExercise}

    \begin{probExercise}
        Show that

        \begin{align}
            \log p(\mathbf{w}|\boldsymbol{t})&=K-\frac{\beta}{2}||\mathbf{t}-\boldsymbol{\Phi}\mathbf{w}||_2^2-\frac{\alpha}{2}||\mathbf{w}||_2^2
        \end{align}

        Therefore, the maximum-a-posteriori parameters of the basis function
        linear regression model are the solution of the regularised
        least-squares problem with $\lambda=\alpha/\beta$.

        Note that, as we will show next, Bayesian linear regression uses the
        full posterior of the parameters to make predictions or to do model
        selection, and not just the maximum-a-posteriori parameters.

    \end{probExercise}
    \normalsize

\end{frame}

\begin{frame}
    \frametitle{Batch Bayesian linear regression: demo code}
    Available \href{https://joacorapela.github.io/gcnuBridging2023/auto\_examples/bayesianLinearRegression/plotBatchBayesianLinearRegression.html\#sphx-glr-auto-examples-bayesianlinearregression-plotbatchbayesianlinearregression-py}{here}
\end{frame}

\subsubsection{Online Bayesian linear regression}

\begin{frame}
    \frametitle{Online Bayesian linear regression: recursive update of posterior distribution of parameters}
	\scriptsize
	\begin{claim}[recursive update]
		If the observations, $\{\mathbf{t}_1,\ldots,\mathbf{t}_n,\dots\}$, are linearly independent when conditioned on the model parameters, $\boldsymbol{\theta}$, then for any $n\in\mathbb{N}$
		\begin{align}
			p(\boldsymbol{\theta}|\mathbf{t}_1,\ldots,\mathbf{t}_n)=K\ p(\mathbf{t}_n|\boldsymbol{\theta})p(\boldsymbol{\theta}|\mathbf{t}_1,\ldots,\mathbf{t}_{n-1})
		\end{align}
		where $K$ is a quantity that does not depend on $\boldsymbol{\theta}$.
	\end{claim}
	\normalsize
\end{frame}

\begin{frame}
    \frametitle{Online Bayesian linear regression: recursive update of posterior distribution of parameters}
	\tiny
	\begin{proof}
		By induction on $H_n: p(\boldsymbol{\theta}|\mathbf{t}_1,\ldots,\mathbf{t}_n)=K\ p(\mathbf{t}_n|\boldsymbol{\theta})p(\boldsymbol{\theta}|\mathbf{t}_1,\ldots,\mathbf{t}_{n-1})$.
		\begin{description}
			\item[$H_1$]
				\begin{align*}
					p(\boldsymbol{\theta}|\mathbf{t}_1)=\frac{p(\boldsymbol{\theta},\mathbf{t}_1)}{p(\mathbf{t}_1)}=\frac{p(\mathbf{t}_1|\boldsymbol{\theta})p(\boldsymbol{\theta})}{p(\mathbf{t}_1)}=K\ p(\mathbf{t}_1|\boldsymbol{\theta})p(\boldsymbol{\theta})
				\end{align*}
			\item[$H_n\rightarrow H_{n+1}$]
				\begin{align*}
					p(\boldsymbol{\theta}|\mathbf{t}_1,\ldots,\mathbf{t}_{n+1})&=\frac{p(\boldsymbol{\theta},\mathbf{t}_1,\ldots,\mathbf{t}_{n+1})}{p(\mathbf{t}_1,\ldots,\mathbf{t}_{n+1})}\\
                                                                               &=\frac{p(\mathbf{t}_{n+1}|\boldsymbol{\theta},\mathbf{t}_1,\ldots,\mathbf{t}_n)p(\boldsymbol{\theta},\mathbf{t}_1,\ldots,\mathbf{t}_n)}{p(\mathbf{t}_1\ldots,\mathbf{t}_{n+1})}\\
                                                                               &=\frac{p(\mathbf{t}_{n+1}|\boldsymbol{\theta})p(\boldsymbol{\theta},\mathbf{t}_1,\ldots,\mathbf{t}_n)}{p(\mathbf{t}_1\ldots,\mathbf{t}_{n+1})}\\
                                                                               &=\frac{p(\mathbf{t}_{n+1}|\boldsymbol{\theta})p(\boldsymbol{\theta}|\mathbf{t}_1,\ldots,\mathbf{t}_n)p(\mathbf{t}_1,\ldots,\mathbf{t}_n)}{p(\mathbf{t}_1\ldots,\mathbf{t}_{n+1})}\\
                                                                               &=K\ p(\mathbf{t}_{n+1}|\boldsymbol{\theta})p(\boldsymbol{\theta}|\mathbf{t}_1,\ldots,\mathbf{t}_n)
				\end{align*}
				Note: the third equality above holds because the observations are assumed to be conditional independent given the parameters.
		\end{description}
	\end{proof}
	\normalsize
\end{frame}

\begin{frame}
    \frametitle{References}

    \tiny{
        \bibliographystyle{apalike}
        \bibliography{probability,informationTheory,machineLearning,gaussianProcesses,latentsVariablesModels,linearDynamicalSystems,numericalMethods}
    }
\end{frame}

\end{document}

    \end{probExercise}
    We integrate Eq.~\ref{eq:modelComparison} using the expression for the
    marginal of the linear Gaussian model,
    Eq.~\ref{eq:marginalLinearGaussianModel}, obtaining

    \begin{align}
        p(\mathbf{t}|\alpha,\beta)=\mathcal{N}(\mathbf{t}|\mathbf{0},\alpha^{-1}\boldsymbol{\phi}\boldsymbol{\Phi}^\intercal+\beta^{-1}\mathbf{I}_N)
    \end{align}

    \normalsize
\end{frame}

\begin{frame}
    \frametitle{References}

    \tiny{
        \bibliographystyle{apalike}
        \bibliography{probability,informationTheory,machineLearning,gaussianProcesses,latentsVariablesModels,linearDynamicalSystems,numericalMethods}
    }
\end{frame}

\end{document}



\section{Bonsai-Python Integration}

\begin{frame}
    \frametitle{Bonsai-Python Integration: Fundamental Concepts}

    \begin{itemize}
        \item architecture
        \item async/sync regimes
    \end{itemize}

\end{frame}

\begin{frame}
    \frametitle{Bonsai-Python Integration: Practical}

    \begin{itemize}
        \item Bonsai-Python Hello World
        \item Bonsai-Python advanced example
    \end{itemize}

\end{frame}

\section{Linear Dynamical Systems}

\begin{frame}
    \frametitle{Linear Dynamical Systems: Fundamental Concepts}

    Fundamental concepts of LDS.

\end{frame}

\begin{frame}
    \frametitle{Linear Dynamical Systems: Practical}

    Estimation of foraging mice kinematics.

\end{frame}

\section{Hidden Markov Models}

\begin{frame}
    \frametitle{Hidden Markov Models: Fundamental Concepts}

    Fundamental concepts of HMMs.

\end{frame}

\begin{frame}
    \frametitle{Hidden Markov Models: Practical}

    Estimation of behavioral states of foraging mice.

\end{frame}

\section{State-Space Decoders}

\begin{frame}
    \frametitle{State-Space Decoders: Fundamental Concepts}

    Fundamental concepts of state-space decoders.

\end{frame}

\begin{frame}
    \frametitle{State-Space Decoders: Practical}

    Mice position decoding from hippocampal population activity.

\end{frame}

\section{Roadmap}

\section{Discussion}

\begin{frame}
    \frametitle{Discussion}

    \begin{itemize}

        \item how to disseminate Bonsai.ML?
            \begin{itemize}

                \item distribute methods, with high-quality code,
                    documentation and examples.

                \item publish papers.

                \item collaborate with experimentalists to use Bonsai.ML
                    to tackle interesting neuroscience problems.

                \item train the trainers.

                \item train Bonsai users on basic machine learning topics.

                \item find a killer Bonsai.ML application.

            \end{itemize}

        \item suggestions for Bonsai.ML roadmap?

        \item suggestions for:

            \begin{itemize}
                \item new ML models to integrate into Bonsai
                \item new applications areas to investigate with Bonsai.ML

                    \begin{itemize}

                        \item testing causality of brain activity patterns on
                            behaviour

                        \item online selection of data to save (e.g. cameras)

                        \item Bonsai for online data analysis (e.g., Terry
                            Sejnowski: for very large datasets, retrieval could
                            be very onerous, so you'd better analyze data
                            online).
                    \end{itemize}

            \end{itemize}

    \end{itemize}

\end{frame}

\begin{frame}
    \frametitle{References}

    \tiny{
        \bibliographystyle{apalike}
        \bibliography{probability,informationTheory,machineLearning,gaussianProcesses,latentsVariablesModels,linearDynamicalSystems,numericalMethods,travelingWaves,receptiveFields,eeg,epilepsy}
    }
\end{frame}

\end{document}
